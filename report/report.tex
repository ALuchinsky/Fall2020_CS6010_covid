\documentclass[conference, onecolumn]{IEEEtran}
\IEEEoverridecommandlockouts
% The preceding line is only needed to identify funding in the first footnote. If that is unneeded, please comment it out.
\usepackage{cite}
\usepackage{amsmath,amssymb,amsfonts}
\usepackage{algorithmic}
\usepackage{graphicx}
\usepackage{textcomp}
\usepackage{xcolor}
\usepackage{blindtext}
\usepackage[hidelinks]{hyperref}
\usepackage{booktabs}
\def\BibTeX{{\rm B\kern-.05em{\sc i\kern-.025em b}\kern-.08em
    T\kern-.1667em\lower.7ex\hbox{E}\kern-.125emX}}
\begin{document}

\title{COVID-19}
\author{Alexey Luchinsky \and Michael Terry  \and Vagish Vela}
\maketitle

\begin{abstract}
  \blindtext
\end{abstract}

\section{Introduction}

\blindtext

\blindtext

\section{Background}

\blindtext 

\begin{eqnarray}
  \label{eq:1}
  e^{i\alpha} = \cos\alpha + i \sin\alpha
\end{eqnarray}

\blindtext 

\blindtext 

\section{Methodology}

\subsection{Ohio}
\label{Ohio}

Ohio state has a good habbit in publishing clean, tidy and very informative data. The main dataset that we will be working with, is called "COVID Summary Data" and can be found following the link \cite{ODRSsummary}.

	This dataset at the current moment has more
        than 128,000 rows and describes all the COVID- related cases in Ohio since Jan 2,2020 till the current date (Nov 18 2020). It should be noted that the dataset represents almost raw data and is in the long-table format. In total there are 9 columns in the table, representing the following fields:
        \begin{itemize}
        \item County --- County of residence
        \item Age Range --- Sex
        \item Onset Date --- Date the illness began. If onset date is unknown, the date associated with the case is used as a substitute for the date of illness onset.
        \item Date of Death --- Date of death. If date of death is unknown, this variable will be listed as “Unknown”.
        \item Admission Date --- Date of hospital admission. If the date is unknown, it will be listed as “Unknown”.
        \item Case count --- Total number of cases that meet the demographic criteria specified in the corresponding row. For example, the number in this cell is the total number of cases in a county, of the given gender, age range, onset date, etc. Each individual is only counted once. This includes both Confirmed or CDC Expanded Case Definition (Probable).
        \item Deaths Due to Illness Count --- Total number of deaths due to illness that meet the criteria in the given row. Each individual is counted once in this dataset. Note: If this cell is “0” and there is a date listed under “Date of Death” this indicates a death that was not considered to be COVID-19 related.
        \item Hospitalized Count --- Sum of hospitalizations that meet the criteria in the given row. Each individual is only counted once in this dataset. Data in this cell is cumulative and not the amount of people that are currently hospitalized.
        \end{itemize}
Typical extract from the dataset is shown on table \ref{tab:OhioDS}.


\begin{table*}
  \tiny
  \centering
  \begin{tabular}{lllllllrrr}
\toprule
{} &    County &      Sex & Age Range & Onset Date & Date Of Death & Admission Date &  Case Count &  Death Due to Illness Count &  Hospitalized Count \\
\midrule
49384  &  Hamilton &   Female &       80+ & 2020-11-19 &           NaN &            NaN &           1 &                           0 &                   0 \\
7791   &    Butler &   Female &      0-19 & 2020-11-19 &           NaN &            NaN &           1 &                           0 &                   0 \\
51824  &  Hamilton &     Male &       80+ & 2020-11-19 &           NaN &            NaN &           2 &                           0 &                   0 \\
10993  &    Butler &     Male &       80+ & 2020-11-19 &           NaN &            NaN &           1 &                           0 &                   0 \\
7588   &     Brown &     Male &       80+ & 2020-11-19 &           NaN &            NaN &           1 &                           0 &                   0 \\
7538   &     Brown &     Male &     60-69 & 2020-11-19 &           NaN &            NaN &           1 &                           0 &                   0 \\
7368   &     Brown &     Male &     20-29 & 2020-11-19 &           NaN &            NaN &           1 &                           0 &                   0 \\
7403   &     Brown &     Male &     30-39 & 2020-11-19 &           NaN &            NaN &           1 &                           0 &                   0 \\
14057  &     Clark &     Male &     70-79 & 2020-11-19 &    10/10/2020 &            NaN &           1 &                           1 &                   0 \\
121524 &     Wayne &  Unknown &      0-19 & 2020-11-19 &           NaN &            NaN &           1 &                           0 &                   0 \\
\bottomrule
\end{tabular}
\caption{Extract from Ohio dataset}
\label{tab:OhioDS}
\end{table*}

The dataset is clean. In table \ref{tab:Ohio:missing} number and percentage of the missing data in each row is shown. As you can see, we have only absent data in "admission Date" and "Date of Deaths" columns (81\% and 95\% respectively). This is not the problem of the dataset and means only that in the majority of the cases the patient was not hospitalized (and the admission date is left blank) and/or is alive (this correspond to undefined "Date of Deaths" field).

\begin{table*}[b]
  \centering
  \begin{tabular}{lll}
    \toprule
    Field Name & Missing (\%) & Comment\\
    \midrule\\
    County & 0 & \\
    Sex & 0 & \\
    Age Range & 0 & \\
               Onset Date & 0 & \\
               Date Of Death & 95 & People with missing Date of Death are still alive \\
               Admission Date &  82 & People with missing Admission Date are not hospitalized \\
               Case Count & 0 & \\
               Death Due to Illness Count & 0& \\
               Hospitalized Count & 0 & \\
    \bottomrule
  \end{tabular}
  \caption{Missing data in Ohio dataset}
  \label{tab:Ohio:missing}
\end{table*}


It was mentioned already, that the dataset is in long-table format, so a lot of information can be extracted from it after some work. For example, it is easy to construct time distributions of the total COVID cases/deaths/hospitalizations, determine the mean time spent in hospital, percentage of people died in hospitals and at home, etc. The other example of the dataset, published by COVID tracker, is available through the link \cite{COVIDTracker}. The data in this dataset is already processed, so only the projections of the original data (like the one doscussed earlier) on different variables are available. This makes it easier to extract some information (like time distributions of the deaths cases), but some other information (the role of age and gender, for example) is lost completely. Later in our report we will compre the results extracted from these two data sets.

The original goal of our project was to study mortality rate due to COVID and compare if with the number of deaths caused by other reasons (natural, cancer, flue, etc). On the Ohio site \cite{NCHSohio} the yearly statistics for last 4 years is presented with classification by different dearth causes. During our work we extracted this information from the site with the help of python Beautiful Soup package \cite{bs4}. In table the \ref{tab:mortalityStats} current information from this site is shown. Later we have found that the same information for all states is available also in nice CSV format from the link \cite{CDCmortality}. 

\begin{table*}
  \centering
\begin{tabular}{lrrrrr}
\toprule
Year &   2014 &   2015 &   2016 &   2017 &   Mean \\
Cause                             &        &        &        &        &        \\
\midrule
Heart Disease                     &  27000 &  28069 &  27410 &  28008 &  27621 \\
Cancer                            &  25433 &  25396 &  25509 &  25643 &  25495 \\
Accidents                         &   6178 &   6756 &   7999 &   8971 &   7476 \\
Chronic Lower Respiratory Disease &   6765 &   7211 &   7015 &   7312 &   7075 \\
Stroke                            &   5791 &   5945 &   5987 &   6425 &   6037 \\
Alzheimer’s disease               &   4083 &   4643 &   5031 &   5117 &   4718 \\
Drugs                             &   2744 &   3310 &   4329 &   5111 &   3873 \\
Diabetes                          &   3641 &   3645 &   3568 &   3740 &   3648 \\
Flu/Pneumonia                     &   2443 &   2445 &   2187 &   2243 &   2329 \\
Kidney Disease                    &   2002 &   2100 &   2262 &   2237 &   2150 \\
Septicemia                        &   1729 &   1957 &   1995 &   2066 &   1936 \\
\bottomrule
\end{tabular}
\caption{Mortality from different causes \label{tab:mortalityStats}}
\end{table*}

\subsection{Michigan}


\subsection{Indiana}






\blindenumerate[10]

\section{ Results \& Discussion }

The main goal of our project is to study the time dependence of death cases caused by COVID virus pandemic and compare it with the available data about deaths caused by other reasons (such as heart descases, usual influencia, ect). It could be interesting also to make such an analysis for different States or even counties.

Let us start with the general yearly distributions. Unfortunately we failed to find data on natural causes statistics by county, so only state-wide analysis was performed.. It turns out that for all considered by our group states on yearly scale the COVID virus was not the main cause of death. In the case of the Ohio state, for example, it is in the 6th place, just between Alzheimer decease and strokes. (see figure \ref{fig:yearly_deaths} for more details). Almost the same situation is observed for other analyzed by our group states. In all cases heart diseases are the most important death causes, while deaths caused by COVID varus are on 4-6 place.


\begin{figure}
  \centering
  \includegraphics[width=0.9\columnwidth]{figs/yearly_deaths}
  \caption{Comparison}
  \label{fig:yearly_deaths}
\end{figure}


This point is rather interesting and seems to to be very  important. It is clear that because of various precaution measures (such as self isolation, quarantine, lockdown of different industries) the life of the whole world changed dramatically in 2020 year because of COVID a lot of necessary medical procedures were delayed or even canceled, lots of industries were almost bankrupted, etc. Not to mention the most close for us now field, i. e. the education (one can argue about some positive sides of the remote education, but it is definitely different completely from the usual face-to-face process and much more difficult for both students and instructors/professors). Thus, we can say that a lot of money and efforts were lost because of the lockdown during this pandemic and a natural question arises: does it worth it, was is really necessary? As you can see from figure \ref{fig:yearly_deaths}, heart diseases are much more dangerous, but we do not stop the life because of them.



To answer this question it could be interesting to perform a more detailed analysis and study the weekly dependence of the statistics of deaths caused by various deceases. It is clear, that COVID data, collected by our group, gives us a nice opportunity to extract time dependence of death cases with such a granularity (either directly, as for Michigan and Indiana states, or after some obvious transformations, as in the case of raw Ohio data). The same information, actually, in available from COVID-tracker project [a] , but we decided to work with the original data first. In figure \ref{fig:RT_comp_NC} we compare time distributions of the deaths cases extracted from the original data and covid-tracker results. From this figure it is clear that for all states under consideration uncumulative distributions are in good agreement, while in the case of shown in figure \ref{fig:RT_comp_CUM} cumulative results the difference is hardly noticeable.

\begin{figure*}
  \centering
  \includegraphics[width=0.9\textwidth]{figs/raw_tracker_comp_nc}
  \caption{Comparison}
  \label{fig:RT_comp_NC}
\end{figure*}

\begin{figure*}
  \centering
  \includegraphics[width=0.9\textwidth]{figs/raw_tracker_comp_cum}
  \caption{Comparison}
  \label{fig:RT_comp_CUM}
\end{figure*}

\begin{figure*}
  \centering
  \includegraphics[width=0.9\textwidth]{figs/weekly_deaths}
  \caption{Comparison}
  \label{fig:weekly_deaths}
\end{figure*}



You can see a typical photo of our hero on figure \ref{fig}.

\begin{figure}[htbp]
\centerline{\includegraphics[width = 0.9\columnwidth]{covid19.png}}
\caption{This is it}
\label{fig}
\end{figure}


\section{Conclusion}

As it was shown in the article \cite{IEEEexample:article_typical}, the night is dark and full of terrors


\bibliographystyle{IEEEtran}
\bibliography{report_litr}

\end{document}

